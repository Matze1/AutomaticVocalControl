\chapter{Introduction}
\label{chapter:introduction}

\section{Motivation}

When a sound engineer edits a song he wants all the recorded audio tracks to be perceptible in the final mix (apart from some special cases). It is most important for audio tracks with notably significance for the musical piece. In this thesis I will work with vocal tracks due to their great significance in meaning, main melody or recognition value of the song. The difficulty with vocal tracks in the mix is the wide dynamic range that singers often use, unlike for example an distorted electric guitar which mainly stays on the same loudness level and is therefore easy to mix with great presence. Almost in every mix the vocals pass through an compressor to reduce their dynamic range. But this is rarely sufficient as compressors are working comparatively fast - too fast to compensate whole song parts  or even some seconds with different vocal levels. For instance when a singer is changing his singing style or he sings instinctively quieter during an instrumental break which may not fit the mix. As a result it is a common procedure to automate an applied gain for every vocal track in the digital audio workstation (DAW) via sketching a gain curve by hand. Obviously this is a time consuming and monotonous task and therefore perfect to hand over to a machine.\\

\section{Idea}

The idea was to write a DAW plug-in that handles the former described problem. A plugin that will sketch a gain curve for a vocal track in real time. This will save the engineer time at every mixing session and the outcome will be more accurate as his own handwork.\\
As the recorded sound pressure level is not transferable to the perceived loudness, my goal was to adapt some algorithmic features of the ITU-R BS.1770-4\footnote{International Telecommunication Union, Recommendation ITU-R BS.1770-4 (10/2015)} “algorithm to measure audio ... loudness”. This algorithm is an up-to-date standard for adjusting audio files to a same humanly perceived loudness level. This will help the gained audio track to stand out in the mix at every point.\\
One plug-in with similar purpose called “Vocal Rider”\footnote{waves.com/plugins/vocal-rider} was published by Waves. 
My aim is to build my plug-in with features I consider important and useful for its purpose, while i do not know about the algorithm behind the Waves plug-in. Hence I will compare the resulting audio files after adapting the gain individually through the “Vocal Rider” and my own plug-in.\\


