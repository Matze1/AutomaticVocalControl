\chapter{Introduction}
\label{chapter:introduction}

\section{Motivation}

When a sound engineer edits a song he wants all the recorded audio tracks to be perceptible in the final mix (apart from some special cases). Most important is this for audio tracks with notably significance for the musical piece. In this thesis we will deal with vocal tracks which often have a significance for the meaning, main melody or recognition value of the song. The problem with vocal tracks in the mix is the wide dynamic range that singers often use, unlike for example an distorted electric guitar which mainly stays on the same loudness level and is therefore easy to mix with great presence. Almost in every mix the vocals path through an compressor to reduce their dynamic range. But this is rarely sufficient as compressors are working comparatively fast. To fast to compensate hole song parts with a different vocal level or even some seconds. For example when a singer is changing his singing style or he sings instinctively quieter during an instrumental break which may not fit the mix. Because of this it is an common procedure to automate a applied gain for every vocal track in the digital audio workstation (DAW) via sketching a gain curve by hand. Obviously this is a time consuming and monotonous task but perfect to hand over to a machine.\\

\section{Idea}

The idea was to write a DAW plug-in that handles the former described problem. A plugin that will sketch a gain curve for a vocal track in real time. This will save the engineer time at every mixing session and the outcome will probably be more accurate as he could do it.\\
As the recorded sound pressure level is not transferable to the perceived loudness, the plan was to adapt some algorithmic features of the ITU-R BS.1770-4 [FOOTNOTE pls] “algorithm to measure audio loudness”. This algorithm is an up-to-date standard for adjusting audio files to a same humanly perceived loudness level. This should help the gained audio track to stand out in the mix at every point.\\
There was published one plug-in with similar purpose called “Vocal Rider” by Waves [fooznotr]. 
I will build my plug-in for this thesis with features i consider important and useful for its purpose, while i do not know about the algorithm behind the Waves plug-in. Hence i will compare the resulting audio files after adapting the gain individually through the “Vocal Rider” and my own plug-in. Further i will evaluate the outcome and verify it with a listening test on independent participates.\\


