\chapter{Introduction}
\label{chapter:introduction}

\section{Motivation}

When a sound engineer is editing a song he/she wants all recorded audio tracks to be perceptible in the final mix (apart from some exceptional cases). This is most important for audio tracks with notably significance for the musical piece. This thesis will deal with vocal tracks due to their meaning, main melody or recognition value of the song. The challenge for vocal tracks in the mix is the wide dynamic range that singers often use unlike for example a distorted electric guitar which mainly stays on the same loudness level and is therefore easy to mix with great presence. In almost every mix the vocals pass through an compressor to reduce their dynamic range. The result of this procedure is rarely satisfactory as compressors are working comparatively fast - too fast to compensate whole song parts or even some seconds of a different remaining vocal level, e.g., when a singer is changing his singing style or is singing quieter during an instrumental break which may not fit the mix. As a result it is a common procedure to automate an applied gain for every vocal track in the digital audio workstation (DAW) via sketching a gain curve by hand. This is a time consuming and monotonous task and therefore predestined to hand over to a machine.\\
Admittedly, professional studio singers could be able to reduce the unwanted dynamics as well but they will rarely succeed completely for the following reasons: As they are covering different singing styles from whispering to screaming it is hard to maintain in perfect proportion to the current backtrack. Additionally this could also effects the current recording setup which can be or has to change between takes (at least pre-gain changes will effect the resulting dynamics). Often a track is still changing its composition and volume level after the vocal recordings as the producer adds or removes certain tracks or effects to get a better result. And finally at the current time of home recordings, where it just needs a single computer to be able to do professional audio mixing, it is not to expect that the majority of vocal recordings are from professional studio singers. Consequently the problem described will occur in many mixing sessions.\\

\section{Objective}

The idea of this study was to develop a DAW plug-in handling the described problem automatically: A plug-in that will sketch a gain curve for a vocal track in real time. This saves time during every mixing session and the outcome would be more accurate than doing the work manually. Furthermore, as an editing tool it should not add colouring to its processed signal, in this aspect differing from many compressors. This is important as the plug-in's main purpose is not a creative task but only to be a tool designed for relieving a mixing engineer of dull work.\\
As the recorded sound pressure level is not transferable to the perceived loudness, the goal was to adapt selected algorithmic features of ITU-R BS.1770-4\cite{ITUalgo}. This algorithm is a standard for adjusting audio files to a same humanly perceived loudness level. This will help the gained audio track to stand out in the mix at every part of the song.\\
A plug-in with similar purpose titled “Vocal Rider”\cite{VR} was published by Waves\footnote{Waves Audio Ltd. is company which is developing software audio effects for professional digital audio signal processing}, but the algorithm behind this plug-in is not published. One aim of this study is to build a plug-in with features considered important and useful for the purpose described and to compare the resulting audio files in order to validate the algorithmic approach and extend the initial idea. In best case this will result in algorithmic progression for this so far rarely filled section of audio editing.\\
The main goal of this study is to enable the plug-in to affect long term dynamics on a single vocal track in real-time. In this case real-time means that the study plug-in should be able to work without any offline calculations. It is still allowed to have some samples of delay at the output, which can be compensated by a DAW. Therefore the plug-in could be used as a insert effect for a DAW track and would react on level changes at playback. As extended goal it is anticipated that the plug-in could additionally receive information about the backtrack via a side chain input and therefore be able to adjust the average loudness of the vocals in relation to the instrumental track.\\

