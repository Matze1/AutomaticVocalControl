\chapter{Introduction}
\label{chapter:introduction}

\section{Motivation}

When a sound engineer edits a song he wants all the recorded audio tracks to be perceptible in the final mix, apart from some special cases. Most important is this for audio tracks with notably significance for the musical piece. In this thesis we will look at vocal tracks which often have a significance for the meaning, main melody or recognition value of the song. The problem with vocal tracks in the mix is the wide dynamic range that singers often use unlike for example an distorted electric guitar which mainly stays on the same sound pressure level and is therefore easy to mix with great presence. Almost in every mix the vocal tracks path through an compressor to reduce the dynamic range. But this is rarely sufficient as compressors are working comparatively fast. To fast to compensate hole song parts with a different vocal level or even some seconds. For example when a singer is changing his singing style or he sings instantly quieter during an instrumental break which may not fit the mix. That is why it is an common procedure to automate a applied gain for every vocal track in the digital audio workstation (DAW) via sketching a gain curve by hand. Obviously this is a time consuming and monotonous task but perfect to hand over to a machine.

\section{Idea}



