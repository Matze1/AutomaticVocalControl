\chapter{Evaluation of the Side Chain Feature}
\label{chapter:evaluation}

The plug-in got some improvements over the time. One significant improvement was the side chain feature which will compensate for a whole further work step of the mixing engineer when it is functioning as intended. To proof the advantages of this feature a listening test was done after the implementation. In the best case this could proof that there is no critical difference to a track volume automation additional to the main plug-in, drawn by a mixing engineer who knows about the level changes in the backtrack. At least it should verify the feature by demonstrate its advantages to the prototype in different circumstances.\\

\section{Test conditions}

To receive results fitting the question about side chain advantages at the performed test, participants had to compare the backtrack/vocal level relation for audio files with the plug-in effecting the vocal gain with and without the side chain feature enabled. Additionally at every comparison there was a reference track where gain automations were drawn with oracle knowledge about the level change of the according backtrack. This track also had the plug-in inserted but without the side chain feature supporting. To establish the equivalence of the test scores an forth audio track with intensional divergent backtrack/vocal relation was added to perform as an anchor.\\
In the first 10 test sections the participants had to listen the reference track as aspired result and subsequently compare the backtrack/vocal level relation with four test items. The test items were mixes of the same song snippet but with varying gain adaption on the vocal tracks: the anchor with intensional divergent gain, a mix with the plug-in while the side chain feature was enabled, a mix altered by the plug-in without a side chain input and an unmodified copy of the reference. While comparing these test items with the as reference declared first audio file the participants did not know about the differences in the creation of the other four. The task was to rank the test items in their similarity of backtrack/vocal level relation to the reference on a percentage scale.\\
As the main plug-in is designed to push comprehensibility and clarity of the vocals in the mix, it remained the question if an additional gain from the side chain feature could influence this in a bad way. Consequently four additional test sections appended the procedure where comprehensibility and clarity where rated by the participants.\\
For the realisation of a valid test a important thing to do was picking usable song parts for the comparison. Therefore three different multitrack projects (see 2.3) were used as source for the audio snippets. As the benefit of the side chain feature is the adaption on backtrack level changes, at the beginning of all three songs the study plug-ins were initialised with settings to fit the local backtrack level. In the middle of the musical piece the backtrack level was changed due to a level automation. This was done in addition to the regular level changes of the backtrack to intensify the differences of the varying plug-in setting to be tested. The reference vocals with oracle knowledge got the same automations as the backtrack and where supplementary levelled to fit the backtrack at all the different song parts. The study plug-in with the side chain feature enabled had to adapt to the backtrack level changes itself while the same plug-in operated on the third vocal variation without the additional side chain informations. The level of the anchor was set manually in order to not resemble the reference track.\\
To have a wide range of circumstances tested for the plug-ins three to four different snippets were taken from each of the songs. These snippets are containing song parts before and after the backtrack level automation just like clips from the exact part where the level change is happening. Additionally the test was extended by snippets of song parts with vocals during instrumental breaks as these circumstances were especially difficult to handle at side chain feature development.\\
While the backtrack/vocal level relations were tested with all different kinds of snippets, the comprehensibility and clarity comparison was only made with clips where the plug-in with the disabled side chain feature was adjusted in its output gain according to the current backtrack level to have a fair comparison.\\
The test was realised in a HTML5 JavaScript browser application build with beaqlejs-0.2\footnote{S. Kraft, U. Zölzer: "BeaqleJS: HTML5 and JavaScript based Framework for the Subjective Evaluation of Audio Quality", Linux Audio Conference, 2014, Karlsruhe, Germany} using the MUSHRA test class. This framework already contained the necessary evaluation sliders and the functionality for the playback of the test items. To avoid confusion about comparison criteria the randomisation of test sections is disabled. Therefore the 10 backtrack/vocal level relation tests are followed by four comprehensibility and clarity tests. In order to receive independent results on each test the test items for evaluation are at randomised positions differing for each of the tests. During the tests the participants were listening to the audio clips on professional studio equipment in a quiet room for unadulterated results.\\
The outcome was transferred into a table and subsequently evaluated.\\