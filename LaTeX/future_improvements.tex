\chapter{Future Improvements}
\label{chapter:future_improvements}

The study plug-in meets the requirement of the objective but it could still be improved.\\
For example, parameter detection is an optional feature which could be useful to set the plug-in correctly, especially at first interactions of a user. As described in chapter \ref{chapter:improvements}.2 and \ref{chapter:improvements}.3, it is only possible to calculate parameter values ‘online’, i.e. during playback of the song. Offline calculation is not implemented yet which would be significantly faster. As one purpose of the plug-in is to save time for the mixing engineer, it would be helpful to enhance the feature with offline calculations in future development. When offline calculation would be available it would be possible to measure the perceived signal level by using the loudness algorithm\cite{ITUalgo} which may further improve the plug-in’s outcome.\\
The plug-in supports stereo input but processes the channels independently. For stereo recorded vocals, e.g. background choirs, it would be advantageous to have the option of linked calculations and equal gain adaption.\\
Another reasonable future improvement would be to extend the scope of application. Currently it is designed to work with vocal signals but other recordings may show similar problems e.g. recordings of the bass guitar. Therefore it would be useful to have the option to chose between different presets for the main constants of the plug-in’s calculation, fitting the different signals.\\
Additional to possible algorithmic improvements the plug-in could be amended based on expanded testing for increased stability and correctness of the results.\\


