\chapter{Future Improvements}
\label{chapter:future_improvements}

The study plug-in meets the requirement of the objective but it could still be improved.\\
For example, parameter detection is an optional feature which could be useful to set the plug-in correctly, especially at first interactions of a user but as described in the corresponding chapter \ref{chapter:improvements}.2 and \ref{chapter:improvements}.3, it is only able to calculate ‘online’ during playback of the song. An offline calculation is not implemented yet which would be significantly faster. As one purpose of the plug-in is to save time for the mixing engineer, it would be great to enhance the feature with offline calculations in future development. When the offline calculation is possible it would be able to measure the perceived signal level by using the loudness algorithm\cite{ITUalgo} which may improve the plug-in’s outcome.\\
An other option of a reasonable future improvement would be to extend its scope of application. Currently it is designed to work with vocal signals but other recordings may show similar problems e.g. recordings of the bass guitar. Therefore it would be useful to have the opportunity to chose between different presets for the main constants of the plug-in’s calculation, fitting the different signals.\\
Additional to possible algorithmic improvements the plug-in could be improved by further testing for increased stability and correctness of the results.\\


