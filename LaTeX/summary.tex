\chapter{Summary}
\label{chapter:summary}

The objective of this study was to develop a plug-in for digital audio workstations which reduces long term dynamics of vocal signals in an automatic way to reduce manual work of mixing engineers.\\
The project started with the creation of a prototype which had the intended functionality implemented: A filter section for simplified imitation of human loudness perception introduces the algorithm of the prototype. Subsequently, a root mean square (RMS) average of the input signal is calculated. A following gate filters audio samples below an automatically adjusted threshold to disable the gain adaption for signal sections without vocals. After passing this gate the difference between loudness goal and current RMS average is determined. This difference is smoothed within a predefined time window and thus results in the final gain. Time window length is changing in dependency of a increasing or decreasing gain adaption (compression or amplification of signal). As last step, the determined gain is multiplied with the input signal.\\
The first version of the prototype was improved by adding additional features: Writing a gain automation into the DAW was made possible as well as holding the gain adaption in place for small breathing gaps within vocal signals. As option automatic parameter settings were implemented for loudness goal and side chain input-gain.\\
As further amendment to these improvements, a side chain feature was added extending the use of the plug-in by adjusting the vocal signal output level to the side chain input level (in anticipated use: the instrumental backtrack).\\
The prototype's outcome was compared to the commercially available plug-in “Vocal Rider”(Waves Audio Ltd., Knoxville, TN, USA) which is a plug-in with similar functions. For this purpose the study algorithm was implemented in Python and the parameters were algorithmically optimized for the smallest possible deviation between outcome of the two plug-ins. The result of the comparison lead to further improvements of the prototype function and of final parameter settings.\\
The final version of the plug-in was realized in the JUCE framework, which turned out as a good choice: It provided basic but solid functionality to facilitate development of an audio plug-in, which otherwise would have been build up from scratch in a time-consuming manner.\\
Finally, the advantages of the side chain feature were evaluated by hearing tests with independent participants indicating that the prototype performed substantially better with enabled side chain feature.\\
\\
\\
In conclusion, a plug-in prototype for digital audio workstations facilitating automatic decreasement of long term dynamics for vocal signals in music production was developed. During the study the plug-in prototype was continuously improved and evaluated in terms of features, algorithms and parameter settings. Thus, the study has lead to a product ready to fulfil its intended task in daily music production. Nevertheless, options for further improvement were identified and described.\\



