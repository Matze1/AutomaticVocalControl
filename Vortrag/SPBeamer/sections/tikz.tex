\section{TikZ Extensions}

\sectionPage{TikZ Extensions}

\begin{frame}[c]{TikZ Extensions}{Overlay Effects in TikZ Images}
    \centering
    \begin{tikzpicture}[thick]
        \draw[visible on=<1->, onslide=<5->uhhred, -latex] (0, 0) -- (1, 0);
        \draw[visible on=<2->, onslide=<6->uhhgreen, -latex] (1, 0) -- (1, 1);
        \draw[visible on=<3->, onslide=<7->uhhblue, -latex] (1, 1) -- (0, 1);
        \draw[visible on=<4->, onslide=<8->uhhstone, -latex] (0, 1) -- (0, 0);
    \end{tikzpicture}
    \vspace{2ex}
    \begin{itemize}
        \item<1-> \texttt{visible on=<num-num>}: make node edge visible on that slide
        \item<5-> \texttt{onslide=\{<num-num>style\}}: apply style on given slides
    \end{itemize}
\end{frame}

\begin{frame}[c]{TikZ Extensions}{Standalone Legends}
    \centering
    \begin{tikzpicture}
        \begin{customlegend}[%
                legend entries={%
                    Entry 1,
                    Entry 2,
                    Entry 3 with comments,
                    Entry 4,
                },
                legend columns={2},
                legend cell align={left},
                legend style={draw=none},
            ]
            \addlegendimage{thick, uhhblue};
            \addlegendimage{area legend, thick, draw=none, fill=uhhred};
            \addlegendimage{surf, thick};
            \addlegendimage{dashed, uhhstone, mark=*}
        \end{customlegend}
    \end{tikzpicture}
    \vspace{2ex}
    \begin{itemize}
        \item Custom legends using \texttt{customlegend} environment
    \end{itemize}
\end{frame}

\begin{frame}[c]{TikZ Extensions}{A Sine in TikZ}
    \centering
    \begin{tikzpicture}
        \begin{axis}[
                scale only axis,
                axis x line=middle,
                axis y line=middle,
                xlabel={$t$},
                ylabel={$x_a(t) = A \cos(2\pi f - \theta)$},
                ymin=-2,
                ymax=2,
                width=6cm,
                height=4cm,
                xtick={0},
                ytick={1},
                yticklabels={$A$},
                yticklabel style={right},
            ]
            \addplot[blue, mark=none, domain=-4:4, samples=500, name path=cos] {cos(deg(2*x + 1))};

            \path[name path=yaxis] (axis cs:0, -2) -- (axis cs:0, 2);

            \path [name intersections={of=cos and yaxis,by=E}];

            \node[fill=blue, circle, inner sep=1pt, pin=30:$A\cos(\theta)$] at(E) {};

            \draw (axis cs:-2.07, -0.75) -- (axis cs:-2.07, -1.25);
            \draw (axis cs:1.07, -0.75) -- (axis cs:1.07, -1.25);
            \draw[latex-latex] (axis cs:-2.07, -1.25) -- node[below] {$T = 1/f$} (axis cs:1.07, -1.25);
        \end{axis}
    \end{tikzpicture}
    \vspace{2ex}
    \begin{itemize}
        \item Demo on how to plot a sine in TikZ
    \end{itemize}
\end{frame}

\begin{frame}[c]{TikZ Extensions}{An STFT Framework in TikZ}
    \centering
    \input{graphics/STFT}
\end{frame}

\begin{frame}[c, fragile]{TikZ Extensions}{An STFT Framework in TikZ}
    \begin{itemize}
        \item Some notes on the STFT framework picture
        \item It requires the following files
            \begin{itemize}
                \item \verb|data/STFT.tex|
                \item \verb|graphics/PersonRed.pdf|
                \item \verb|graphics/Monkey.pdf|
            \end{itemize}
        \item Changing the folders of these files requires editing of the graphic itself
        \item The graphic itself is given in \verb|graphics/STFT.tex|
    \end{itemize}
\end{frame}

% \begin{frame}[c]{TikZ Extensions}{Image plots}
%     \centering
%     \begin{tikzpicture}[]
%         \begin{groupplot}[
%                 group style={%
%                     group size={2 by 2},
%                     xlabels at=edge bottom,
%                     ylabels at=edge left,
%                     horizontal sep=2.5cm,
%                 },
%                 axis on top,
%                 scale only axis,
%                 ymin=0,
%                 ymax=4,
%                 xmin=1.75,
%                 xmax=3.5,
%                 colormap name=viridis,
%                 xlabel={$t~/~\text{s}$},
%                 ylabel={$f~/~\text{kHz}$},
%                 width=0.3\linewidth,
%                 height=2.5cm,
%                 colorbar,
%                 colorbar style={width=2ex, xshift=-2ex},
%                 title style={anchor=south, at={(0.5, 0.9)}},
%             ]
%             \nextgroupplot
%             \imagesc{data/spectrogram_noisy.tsv}
% 
%             % Use different color range
%             \nextgroupplot[
%                 title={Different color range},
%                 point meta min=-30,
%                 colormap name=hot2,
%                 point meta max=0]
%             \imagesc{data/spectrogram_noisy.tsv}
%             \node[draw, fill=white, anchor=north west, font=\footnotesize] at (rel axis cs:0, 0.99) {colormap name=hot2};
% 
%             % Show different excerpt
%             \nextgroupplot[
%                 title={Other excerpt},
%                 point meta min=-30,
%                 point meta max=0,
%                 colormap name=parula,
%                 xmin=2.75,
%                 xmax=4.5,
%                 ymin=0,
%                 ymax=8]
%             \imagesc{data/spectrogram_noisy.tsv}
%             \node[draw, fill=white, anchor=north west, font=\footnotesize] at (rel axis cs:0, 0.99) {colormap name=parula};
% 
%             % Use custom defined colormap (it's intentionally ugly)
%             \nextgroupplot[
%                 title={Custom Colormap},
%                 colormap name=temp, 
%                 point meta min=-30,
%                 point meta max=10,
%                 xmin=3.5,
%                 xmax=5.5]
%             \imagesc{data/spectrogram_noisy.tsv}
%             \node[draw, fill=white, anchor=north west, font=\footnotesize] at (rel axis cs:0, 0.99) {colormap name=temp};
%         \end{groupplot}
%     \end{tikzpicture}
% \end{frame}
